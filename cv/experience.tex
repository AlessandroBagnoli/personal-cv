%-------------------------------------------------------------------------------
%	SECTION TITLE
%-------------------------------------------------------------------------------
\cvsection{Esperienze Lavorative}


%-------------------------------------------------------------------------------
%	CONTENT
%-------------------------------------------------------------------------------
\begin{cventries}

\cventry
	{Backend Software Developer} % Job title
	{\href{https://overit.it/}{OverIT}} % Organization
	{Riccione (RN), Italia - Full remote} % Location
	{Ottobre 2021 - Oggi} % Date(s)
	{
		Sviluppo e manutenzione di applicativi web custom lato backend che espongono API REST, con una forte attenzione ai test unitari e funzionali.
		Maggiori tecnologie coinvolte:
		\begin{itemize}
			\item {Java 8}
			\item {Maven come build tool dei progetti}
			\item {Spring Framework}
			\item {Hibernate con Spring Data per la gestione della persistenza}
			\item {JUnit e Mockito per la scrittura di test automatici}
			\item {WildFly come application server}
			\item {ActiveMQ come message broker per la comunicazione asincrona tra diversi moduli applicativi}
			\item {Git con workflow Gitflow per il versionamento}
		\end{itemize}
	}

%---------------------------------------------------------

\cventry
    {Backend Software Developer} % Job title
    {\href{https://i-tel.it}{I-Tel}} % Organization
    {Riccione (RN), Italia} % Location
    {Agosto 2018 - Ottobre 2021} % Date(s)
    {
      	Analisi dei requisiti, sviluppo, deployment e manutenzione del backend di due dei prodotti aziendali:
      	\begin{itemize}
      		\item {Un sistema automatico di richiamata verso il paziente e richiesta di conferma/cancellazione prenotazione.}
      		\item {Un sistema per la comunicazione di presenze, assenze, turni, ferie e permessi da parte dei dipendenti delle aziende clienti tramite diverse canalità quali IVR e applicazioni mobile, con possibilità di integrazione ai sistemi gestionali già in essere presso il cliente.}
      	\end{itemize}
      	Oltre a curare i due prodotti standard, mi sono occupato anche della progettazione ed implementazione di progetti custom e middleware necessari per l'integrazione a sistemi di backend esterni.
      	Nel corso dell'ultimo anno in I-Tel ho iniziato lo studio e i lavori atti alla migrazione di uno dei due prodotti dalla piattaforma di sviluppo proprietaria aziendale, verso una nuova architettura modulare e distribuita, basata su framework noti, come Spring e Hibernate.
      	Maggiori tecnologie coinvolte:
      	\begin{itemize}
      		\item {Java nelle versioni 8 e 11}
      		\item {SVN e Git come sistemi di controllo di versione}
      		\item {PostgreSQL come DBMS relazionale}
      		\item {Tomcat come application server}
      		\item {Asterisk come PBX integrato a centralini/voice gateway per la parte VoIP degli applicativi sviluppati}
      		\item {Linux CentOS/RHEL come sistemi operativi sui quali viene eseguito il deploy dei progetti in produzione.}
      	\end{itemize}
    }

%---------------------------------------------------------

%---------------------------------------------------------

\end{cventries}
