%-------------------------------------------------------------------------------
%	SECTION TITLE
%-------------------------------------------------------------------------------
\cvsection{Esperienza Lavorativa}


%-------------------------------------------------------------------------------
%	CONTENT
%-------------------------------------------------------------------------------
\begin{cventries}

\cventry
	{Backend Software Developer} % Job title
	{\href{https://overit.it/}{OverIT}} % Organization
	{Milano (MI), Italia} % Location
	{Ottobre 2021 - Oggi} % Date(s)
	{
		TBD
	}

%---------------------------------------------------------

\cventry
    {Software Developer} % Job title
    {\href{https://i-tel.it}{I-Tel}} % Organization
    {Riccione (RN), Italia} % Location
    {Agosto 2018 - Ottobre 2021} % Date(s)
    {
      	Attualmente impiegato nella divisione delivery, le mie attività principali riguardano lo sviluppo, deployment ed eventuale manutenzione correttiva/evolutiva del backend di due dei prodotti di punta dell'azienda:
      	\begin{itemize}
      		\item {SmartRecall: sistema automatico di richiamata verso il paziente e richiesta di conferma/cancellazione prenotazione.}
      		\item {SmartC6: sistema per la comunicazione di presenze, assenze, turni, ferie e permessi da parte dei dipendenti delle aziende clienti tramite diverse canalità quali IVR e applicazioni mobile, con possibilità di integrazione ai sistemi gestionali già in essere presso il cliente.}
      	\end{itemize}
      	Oltre a curare i due prodotti standard, mi occupo anche della progettazione ed implementazione di eventuali progetti custom o middleware necessari per l'integrazione a sistemi di backend esterni.
      	Da Maggio 2021 ho iniziato lo studio e i lavori atti alla migrazione del prodotto SmartC6 su una nuova architettura modulare e distribuita, basata su Framework Spring (Boot) e containerizzazione su Docker.
      	Maggiori tecnologie coinvolte:
      	\begin{itemize}
      		\item {Java 8/11 come linguaggio di riferimento}
      		\item {SVN/Git come sistemi di controllo di versione}
      		\item {PostgreSQL/SQLServer/MySQL come DBMS relazionali}
      		\item {Tomcat come application server/contenitore di servlet}
      		\item {Asterisk come PBX integrato a centralini/voice gateway per la parte VoIP degli applicativi sviluppati}
      		\item {Linux CentOS/RHEL come sistemi operativi sui quali viene eseguita l'installazione dei progetti in produzione.}
      	\end{itemize}
    }

%---------------------------------------------------------

%---------------------------------------------------------

\end{cventries}
