%-------------------------------------------------------------------------------
%	SECTION TITLE
%-------------------------------------------------------------------------------
\cvsection{Esperienze Lavorative}


%-------------------------------------------------------------------------------
%	CONTENT
%-------------------------------------------------------------------------------
\begin{cventries}
	
\cventry
{Backend Engineer} % Job title
{\href{https://alpian.com/it}{Alpian}} % Organization
{Riccione (RN), Italia - Full remote} % Location
{Maggio 2022 - Oggi} % Date(s)
{
	Sviluppo lato backend all'interno di uno Scrum Team della prima banca privata digitale Svizzera, sfruttando un'architettura a microservizi e l'approccio cloud native con GCP come provider.
	\begin{itemize}
		\item {Linguaggi e strumenti: Java 11, Kotlin, Go, Maven}
		\item {Frameworks: Spring Boot, Hibernate}
		\item {Databases: relazionali (PostgreSQL) e non relazionali (MongoDB, Google Datastore)}
		\item {Librerie e framework di testing: JUnit, Mockito, Testcontainers}
		\item {Comunicazione: sincrona (REST, gRPC), e asincrona (Google Pub/Sub, ActiveMQ)}
		\item {Versioning: Git con Github-Flow}
	\end{itemize}
}

%---------------------------------------------------------

\cventry
	{Backend Software Developer} % Job title
	{\href{https://overit.it}{OverIT}} % Organization
	{Riccione (RN), Italia - Full remote} % Location
	{Ottobre 2021 - Maggio 2022} % Date(s)
	{
		Sviluppo e manutenzione lato backend di applicazioni web custom con esposizione di API consumate da un BPM enterprise e da frontend web.
		\begin{itemize}
			\item {Linguaggi e strumenti: Java 8, Maven, WildFly}
			\item {Frameworks: Spring Framework, Hibernate}
			\item {Databases: relazionali (PostgreSQL)}
			\item {Librerie e framework di testing: JUnit, Mockito}
			\item {Comunicazione: sincrona (REST), e asincrona (ActiveMQ)}
			\item {Versioning: Git con Git-Flow}
		\end{itemize}
	}

%---------------------------------------------------------

\cventry
    {Backend Software Developer} % Job title
    {\href{https://i-tel.it}{I-Tel}} % Organization
    {Riccione (RN), Italia} % Location
    {Agosto 2018 - Ottobre 2021} % Date(s)
    {
      	Design ed implementazione di progetti custom e middleware necessari per l'integrazione a sistemi di backend di terze parti.
      	Migrazione di uno dei prodotti aziendali da una piattaforma di sviluppo proprietaria ad una nuova architettura modulare, basata su framework noti.
      	\begin{itemize}
      		\item {Linguaggi e strumenti: Java 8/11, Tomcat, Asterisk}
      		\item {Frameworks: Spring Boot, Hibernate}
      		\item {Databases: relazionali (PostgreSQL)}
      		\item {Comunicazione: sincrona (REST, SOAP)}
      		\item {Versioning: Git, SVN}
      	\end{itemize}
    }

%---------------------------------------------------------

\end{cventries}
